
\chapter{Crédibilité de stabilité}
\section{Introduction par un exemple}
\label{chap:intro:ex}
On utilise l'exemple suivant pour introduire la théorie de la crédibilité de stabilité.
Vous travaillez dans une compagnie d'assurance ayant 2 agents:
\begin{itemize}
\item Agent 1 : Ouvert depuis 10 ans;
\item Agent 2 : Nouvellement ouvert cette année.
\end{itemize}
On veut comparer les \textit{taux de succès} $\Big(S = \frac{\text{Nombre de ventes}}{\text{Nombre de soumissions}}\Big)$:

\bigskip
\begin{tabular}{|c|c|c|c|}
  \hline
  Agent & Nombre de soumissions & Nombre de ventes & Taux de succès   \\
  \hline
  1 & 5000 & \numprint{2500} & 50$\%$  \\
  2 & 50 & 25 & 50$\%$ \\
  \hline
\end{tabular}\\
\bigskip

\subsection*{Question principale en crédibilité de stabilité :}
À partir de quelle taille (n) est-ce que $S$ est statistiquement stable ($\Rightarrow$ crédible) ?
\subsection*{Traitement mathématique :}
On dit que la statistique (dans notre cas présent $S$) est \emph{crédible} (ou \emph{statistiquement stable}) à l'ordre (\textit{k, P}) lorsque:
\begin{equation}
\label{sec:eq:normale}
P \Big \lbrace(1 - k) \times E[s] \leq S \leq(1 + k)\times E[s]\Big\rbrace \geq P
\end{equation}
Où \emph{k} est petit (ex.: 5\%) et \emph{P} est près de 100 \% (ex.: 95\%). 
\\

On peut interpréter l'équation \ref{sec:eq:normale} comme étant la probabilité p\% d'être à l'extérieur de notre intervalle autour de la statistique  S. 
\\

Selon notre exemple initial :
\begin{align*}
S &= \frac{\text{Nombre de ventes}}{\text{Nombre de soumissions}} \\
&= \frac{\sum_{i = 1}^{n} I_{i}}{n} \\
& \Rightarrow \Bigg(\frac{\sum_{i = 1}^{n} I_{i}}{n} \Bigg) \sim \text{Bin(n,p)}
\end{align*}
À partir de quelle valeur de n (taille) est-ce que :
\begin{align*}
P \Big \lbrace(1 - k) \times E[s] \leq S \leq(1 + k)\times E[s]\Big\rbrace \geq P
\end{align*}
Puisqu'on s'attend à un n grand, on utilise l'approximation normale. En assumant que le \textit{vrai} taux de succès est de 50 \% et que (k = 5\%, p = 90 \%), l'équation devient:
\begin{align*}
P\Bigg(0.95 \times  0.5 \times  n &\leq \frac{\sum_{i = 1}^{n} I_{i}}{n} \leq 1.05 \times  0.5 \times  n \Bigg) \geq P\\
P\Bigg(0.475 \times  n &\leq \frac{\sum_{i = 1}^{n} I_{i}}{n} \leq 0.525 \times  n\Bigg) \geq0.9 \\
P\Bigg( \frac{0.475n - 0.5n}{\sqrt{n \times  0.5 \times 0.5}} &\leq \frac{\sum_{i = 1}^{n} I_{i}}{n} \leq \frac{0.525n - 0.5n}{\sqrt{n \times  0.5 \times 0.5}}\Bigg) \geq0.9 \\
P \Big \lbrace -0.05 &\sqrt{n} \leq Z \leq 0.05\sqrt{n} \Big \rbrace \geq 0.9 \\
\Phi \big (0.05 \sqrt{n}) &- \Phi\big (-0.05 \sqrt{n}\big) \geq0.9 \\
\Phi\big (0.05 \sqrt{n}\big) &- \Big(1 - \Phi(-0.05 \sqrt{n}\big)\Big) \geq0.9 \\
2 \Phi\big (0.05 \sqrt{n}\big) &- 1 \geq0.9 \\
\Phi\big (0.05 \sqrt{n}\big) & \geq 0.95 \\
0.05 \sqrt{n} & \geq \Phi^{-1}\big (0.95\big) \\
&n \geq 1082.41
\end{align*}
Pour conclure, à partir d'environ $\approx$ 1083  soumissions, on peut \textit{croire} au taux de succès des agents de 50\%.

\section{Crédibilité complète d'ordre $(k,p)$}
En crédibilité complète, un contrat (ou un portefeuille de contrat) est \emph{crédible} si son expérience est stable. Intuitivement, cette stabilité va de pair avec la \emph{taille} du contrat ou du portefeuille.

Ainsi, une crédibilité complète d'ordre $(k,p)$ est attribuée à l'expérience $S$ d'un contrat ou portefeuille si les paramètres de la distribution sont tels que 
\begin{align*}
P\Bigg( \frac{(1 - k)E[S] - E[S]}{\sqrt{Var(S)}} \leq \frac{S - E[S]}{\sqrt{Var(S)}} \leq \frac{(1 + k)E[S] - E[S]}{\sqrt{Var(S)}}\Bigg) \geq P \\
\end{align*}
est vérifié.
\subsection{Exemple classique:}
On considère le modèle classique du risque IARD:
\begin{equation}
	S =
     \left\{
     \begin{array}{rl}
      \sum_{i = 1}^{n} X_{i} &, \text{si } N > 0 \\
      0 &, \text{si }N = 0
     \end{array}
     \right.
\end{equation}
\begin{itemize}
\item[S:] Coûts totaux d'un portefeuille d'assurance IARD pour 1 an
\item[N:] Nombre de sinistres en 1 an $ \sim$Poisson($\lambda$)
\item[$X_i:$] Coût du sinistre \emph{iid} avec $F_X(x),f_X(x), E[X], Var(X)...$
\end{itemize}
On sait que:
\begin{align*}
E[S] &= E[N] E[X] \\
&= \lambda E[X]\\
\end{align*}
ainsi que,
\begin{align*}
Var(S) &= E[N]Var(X) + Var(N)E^{2}[X]\\
&= \lambda Var(X) + \lambda E^{2}[X]\\
&= \lambda (Var(X) + E^{2}[X])\\
&= \lambda E[X^2]
\end{align*} 
On reprend l'équation \emph{générale} \ref{sec:eq:normale} :
\begin{align*}
P \Big \lbrace(1 - k) \times E[s] \leq S \leq(1 + k)\times E[s]\Big\rbrace \geq P
\end{align*}

Par conséquent, on cherche la taille n ($\lambda$) à partir de laquelle les \textit{résultats} $(S)$ sont statistiquement stables (\textit{Also Know As} crédibles) d'année en année. À partir de l'équation \ref{sec:eq:normale}, on effectue une approximation par la loi Normale
\begin{equation}
P\Bigg( \frac{(1 - k)E[S] - E[S]}{\sqrt{Var(S)}} \leq \frac{S - E[S]}{\sqrt{Var(S)}} \leq \frac{(1 + k)E[S] - E[S]}{\sqrt{Var(S)}}\Bigg) \geq P \\
\end{equation}
À partir de la forme générale \ref{sec:eq:normale}, on obtient:
\begin{align*}
P\Bigg( \frac{ k \lambda E[S]}{\sqrt{\lambda E[X^2]}} &\leq Z \leq \frac{k \lambda E[X]}{\sqrt{\lambda E[X^2]}}\Bigg) \geq P \\
2 \Phi\Bigg(k \sqrt{\lambda} &\frac{E[X]}{\sqrt{E[X^2]}}\Bigg) -1 \geq P \\
\Phi \Bigg( k \sqrt{\lambda}& \frac{E[X]}{\sqrt{E[X^2]}}\Bigg) \geq \frac{P + 1}{2}\\
\Bigg(\frac{k \sqrt{\lambda} E[X]}{\sqrt{E[X^2]}} &\Bigg) \geq \Phi^{-1}\Bigg(\frac{P + 1}{2}\Bigg)\\
\lambda \geq  \Bigg(\frac{\sqrt{E[X^2]}}{E[X] k } &\Bigg) \Phi^{-1}\Bigg(\frac{P + 1}{2}\Bigg)^2\\
\lambda  \geq \Bigg(\frac{\Phi^{-1}(\frac{P + 1}{2})^2}{k}& \Bigg) \Bigg(\frac{Var(X) + E^2[X]}{E^2[X]}\Bigg)\\
\end{align*}
On obtient l'équation finale suivante 
\begin{equation}
\lambda  \geq \Bigg(\frac{\Phi^{-1}(\frac{P + 1}{2})^2}{k^2} \Bigg) (1 + CV^2(X))
\end{equation}
Remarques:
\begin{itemize}
\item[1)]Dans ce cas-ci, la \textit{taille} du portefeuille est exprimée en \textit{nombre de sinistres espérés annuellement ($\lambda$)} et non en nombre d'assurés, afin de rendre la théorie portable dans plusieurs lignes d'affaires (auto, habitations, ...).
\item[2)]Plus la distribution de X (sévérité) est volatile (plus CV(X) $\nearrow$) plus la \textit{taille} doit être grande pour atteindre la crédibilité complète.
\item[3)]En assumant:
		\begin{itemize}
		\item k = 5\%
		\item P = 90\%
		\item La sévérité est non aléatoire (densité dégénérée) $\Rightarrow$ CV(X) = 0
		\end{itemize}
		On obtient,
		\begin{align*}
		\lambda &\geq \Bigg(\frac{\Phi^{-1}(0.95)}{0.05}\Bigg)^{2} (1 + 0^2)\\
		&\geq 1082.41
		\end{align*}
\end{itemize}

\section{Crédibilité partielle}
Que fait-on lorsque la \textit{taille} est inférieure au seuil de crédibilité complète?
\begin{itemize}
\item[• Option 1: ]Ne pas considérer les chiffres
\item[• Option 2: ]Whitney (1918) proposa de pondérer l'expérience individuelle avec un complément crédible (ex. prime collective). 
\\
Soit :
\begin{equation}
\label{eq:crédipartie}
\pi = Z \times  \overline{S} + (1 - Z) \times  m
\end{equation}
	\begin{itemize}
	\item $\pi$  est la statistique $S$ crédibiliser
	\item Z est le facteur de pourcentage de crédibilité
	\item $\overline{S}$ est l'expérience observée par la statistique \textit{S}
	\item m est le complément crédible
	\end{itemize}
\end{itemize}
Sans trop développer de théorie, Whitney proposa diverses formes pour Z. 
\\

Z peut correspond à :
\begin{itemize}
\item[i)]
	\begin{align*}
	Z = \text{min} \Bigg( \sqrt{\frac{\text{taille}(n)}{\text{taille de crédibilité complète}(n_0)}} ; 1 \Bigg)
	\end{align*}
\item[ii)]
	\begin{align*}
	Z = \frac{n}{n + k}, \text{ où n = taille, k = constante arbitraite}
	\end{align*}
\end{itemize}
\subsubsection*{Retour sur l'exemple \ref{chap:intro:ex}:}
On choisit la formule (i) pour Z, sachant que $n_0 = 1083$, alors
\begin{align*}
Z &= \text{min} \Bigg( \sqrt{\frac{n}{n_0}} ; 1 \Bigg)\\
&=\text{min} \Bigg( \sqrt{\frac{50}{1083}} ; 1 \Bigg)
\end{align*} 
On suppose que le taux de succès de toutes les agences au Canada est de 45 \%, assez crédible pour être choisi comme \textit{complément}, car l'ensemble du Canada correspond à un échantillon de grande \textit{taille}.

\bigskip
\begin{tabular}{|l|l|c|c|r|r|}
  \hline
  Agent & n & $\sum_{i =1}^{n} X_i$ & S & Z & $\pi$   \\
  \hline
  1 & \numprint{5000} & \numprint{2500} & 50$\%$ & 100\% & 50\% \\
  2 & 50 & 25 & 50$\%$ & 21.5\%  & 46.1\% \\
  \hline
\end{tabular}\\
\bigskip

Remarques : 
\begin{itemize}
\item Si la moyenne nationale avait été plus élevée, disons 55 \%, alors $\pi_2 = 54\%$, donc on donne le \emph{bénéfice du doute} à l'agent parce qu'il est trop peu crédible.
\item Le but de la crédibilité partielle est d'incorporer autant d'expérience individuelle dans la prime, pour bien le faire,il faut déterminer les paramètres adéquatement.
\end{itemize}